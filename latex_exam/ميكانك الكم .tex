\documentclass[a4paper,14pt]{article}
\usepackage[left=2cm,right=2cm,top=2cm,bottom=1.7cm,headsep=5mm]{geometry}
\usepackage{graphicx}
\usepackage{amsmath,amsfonts,amssymb}
\usepackage{enumitem}
\usepackage{fancyhdr}
\usepackage{tikz}
\usetikzlibrary{calc}
\usepackage{siunitx} 
\usepackage{fourier} % math font
\usepackage[normalem]{ulem}  
\usepackage{esvect}
\usepackage{fancyhdr}
\usepackage{polyglossia}

% desactiver message d'erreur bidi package -------------------
\makeatletter 
\AtBeginDocument{\bidi@isloaded[]{arabxetex}}
\makeatother


% languages & fonts===========================================
\setdefaultlanguage[calendar=gregorian,numerals=maghrib]{arabic}
\setotherlanguage{english}
\newfontfamily\arabicfont[Script=Arabic,Scale=1.2]{Amiri}
\newfontfamily\arabicfontsf[Script=Arabic,Scale=1]{Amiri}

%=============================================================

\renewcommand{\footrulewidth}{1pt}
\renewcommand{\headrulewidth}{1pt}

\rhead{{ المدرسة العليا للأساتذة بشار}\LOGO\\}

\lhead{$2024-2023$}
\cfoot{}

\lfoot{\sffamily بالتوفيق}

\parindent 0pt
\setlength{\headsep}{25pt}% defaut 25pt
\setlength{\itemsep}{3cm}
\arraycolsep=1pt

\pagestyle{fancy}


\def\LOGO{%
	\begin{picture}(0,0)\unitlength=0.5cm
		\put (8,-0.9) {\includegraphics[width=4em]{logo_ens.png}}
	\end{picture}
}


\begin{document}
	
\begin{center}
	{\Large\sffamily إمتحــان السداسي الأول في ميكـانيك الكم المعمق }
\end{center}
\rule{\textwidth}{1pt}\\
{\sffamily السنة الخامسة أساتذة التعليم الثانوي}\hspace*{\fill}{\sffamily $ 2024/01/15 $}
	
\subsubsection*{\fbox{\sffamily{\large التمرين الأول: (4ن)}}}		
		
نعتبر الحالة 
\begin{equation*}
	|\psi\rangle= \dfrac{1}{\sqrt{2}}|\phi_{1}\rangle + \dfrac{1}{\sqrt{5}}|\phi_{2}\rangle + \dfrac{1}{\sqrt{10}}|\phi_{3}\rangle
\end{equation*}	

والتي تمثل تراكب ثلاث حالات ذاتية متعامدة	
$|\phi_{1}\rangle,~|\phi_{2}\rangle,~	|\phi_{3}\rangle $ 
للمؤثر $\hat{B}$ حيث أن $\hat{B}|\phi_{n}\rangle = n^{2}|\phi_{n}\rangle$.

\begin{itemize}
	\item 
أوجد القيمة المتوقعة $(\left\langle\hat{B}\right\rangle)$ للمؤثر $\hat{B}$ على الحالة $|\psi\rangle$.
\end{itemize}

\subsubsection*{\fbox{\sffamily{\large التمرين الثاني: (6ن)}}}
	
نعتبر نظامًا فيزيائيًا يرتبط فضاء حالته، بالأساس المتعامد وهو ثلاثي الأبعاد يتكون من الأشعة $|u_{1}\rangle, 	|u_{2}\rangle, 	|u_{3}\rangle $ . المؤثريين $\hat{L_{z}}$ و $\hat{S}$ المعرفين بالعلاقتين:

\begin{equation*}
	\label{eqn:1}
	\begin{array}{ccl}
		\hat{L_{z}}\mid u_{1}\rangle = |u_{1}\rangle,~~ \hat{L_{z}}\mid u_{2}\rangle= 0,~~~~ \hat{L_{z}}\mid u_{3}\rangle=-|u_{3}\rangle\\
		
		\hat{S} \mid u_{1}\rangle = |u_{3}\rangle,~~~ \hat{S}\mid u_{2}\rangle=|u_{2}\rangle , ~~~\hat{S}\mid u_{3}\rangle=|u_{1}\rangle
	\end{array}
\end{equation*} 
 
	\begin{enumerate}
		
		\item  
اكتب المصفوفات التي تمثل المؤثرات $\hat{L_{z}},~~\hat{L^{2}_{z}},~~\hat{S},~~\hat{S^{2}}$ في الأساس $|u_{1}\rangle,~|u_{2}\rangle,~	|u_{3}\rangle $ 

		\item هل هذه المؤثرات ملحوظات (يمكن ملاحظتهم) ؟
		
		
		\item 
أعط شكل المصفوفة الأكثر عمومية التي تمثل المؤثر المتبادل مع المؤثر $\hat{L_{z}}$.
		
		
	\end{enumerate} 
	
\subsubsection*{\fbox{\sffamily{\large التمرين الثالث: (10ن)}}}
	
\begin{itemize}
	\item 
	
	
	\begin{enumerate}
				
		\item 
	
		\item
	
				
		\end{enumerate}	
	\item
	
	
	\begin{enumerate}
		
		\item 
	
		\item
		
		\item
	
	\end{enumerate}	
\item

	\begin{enumerate}
	
	\item 

	\item
	
   \end{enumerate}	

\end{itemize}	

	

\end{document}