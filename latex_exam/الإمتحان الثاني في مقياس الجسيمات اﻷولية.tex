\documentclass[a4paper,14pt]{article}
\usepackage[left=2cm,right=2cm,top=2cm,bottom=1.7cm,headsep=5mm]{geometry}
\usepackage{graphicx}
\usepackage{amsmath,amsfonts,amssymb}
\usepackage{enumitem}
\usepackage{fancyhdr}
\usepackage{tikz}
\usetikzlibrary{calc}
\usepackage{siunitx} 
\usepackage{fourier} % math font
\usepackage[normalem]{ulem}  
\usepackage{esvect}
\usepackage{polyglossia}

% desactiver message d'erreur bidi package -------------------
\makeatletter 
\AtBeginDocument{\bidi@isloaded[]{arabxetex}}
\makeatother


% languages & fonts===========================================
\setdefaultlanguage[calendar=gregorian,numerals=maghrib]{arabic}
\setotherlanguage{english}
\newfontfamily\arabicfont[Script=Arabic,Scale=1.3]{Amiri}
\newfontfamily\arabicfontsf[Script=Arabic,Scale=1.5]{Amiri}

%=============================================================

%\usepackage[novoc]{arabxetex}


\def\LOGO{%
	\begin{picture}(0,0)\unitlength=1cm
		\put (3,-1) {\includegraphics[width=5em]{logo.png}}
	\end{picture}
}

\begin{document}
	\begin{center}
		%\sffamily\bfseries
		{ المدرسة العليا للأساتذة بشار}\LOGO\\
		\begin{flushright}
			قسم السنة الخامسة أساتذة التعليم الثانوي                  ~~~~~~~~~~~~~~~~~~~~~~~~~~~~~~~~~~~~~~~~~~~~~~~~~~~~~~~~~~~~~~~~~~~~~~~~~~ 2022/01/16\\
		\end{flushright}
		إمتحــان السداسي الأول في فيزيـــــاء الجسيمـات 
	\end{center}
	
	\hrulefill\par
	
\subsubsection*{\uline{\sffamily{\large التمرين الأول: (8ن)}}}
	
	ضع كلمة  (صحيح) أو (خطأ) المناسبة أمام كل عبارة؟
	\begin{enumerate}
		\item الجسيمات اﻷولية هي الجسيمات التي لها تركيب داخلي 
		(~~~~~~~)
		\item تعتبر البروتونات جسيمات أولية
		(~~~~~~~)
		\item اللبتونات أحد أقسام الجسيمات اﻷولية تنضوي تحتها الميونات
		(~~~~~~~)
		\item حسب نظرية النموذج العياري تصنف الجسيمات إلى جيلين
		(~~~~~~~)
		\item الجيل اﻷول يتضمن  الكوارك الستة (u,d,c,s,t,b)
		(~~~~~~~)
		\item 	الجيل الثاني يتضمن الكوارك (c,s) و كذلك ميون و نترينو ميون 
		(~~~~~~~)
		\item    حاملات القوى هي الجسيمات اﻷولية المسؤلة عن التفاعلات اﻷساسية
		(~~~~~~~)
		\item الجسيمات و الجسيمات المضادة لا تختلف في الكتلة
		(~~~~~~~)
		\item أكتشفت الكورك الثلاثة ذات الكتل الصغيرة أولا
		(~~~~~~~)
		\item  تصنف الجسيمات إلى مجموعتين بحسب قيمة الدوران مغزلي
		(~~~~~~~)
		\item  الجسيمات التي قيمة دورانها المغزلي أنصاف الأعداد الصحيحة ( 1/2, 3/2, 5/2,… ) تسمى بـالفرميونات
		(~~~~~~~)
		\item تتفاعل الجسيمات مع حقل هيقز مكتسبة بذلك شحنة لونية
		(~~~~~~~)
		\item النموذج العياري النهائي صنف الجسيمات إلى ستة كوارك و ستة ليبتونات و ستة بوزنات 
		(~~~~~~~)
		\item أكبر جسيم من حيث الكتلة هو الكوارك السفلي botom
		(~~~~~~~)
		\item الجسيم المفقود في النموذج العياري الحالي هو ~....... 
		\item يتم تسريع الجسيمات .......... في المصادمات الهيدرونية.
		\item الإحداثي الثابت في الفضاء الرباعي هو .........
		
	\end{enumerate} 
	
\subsubsection*{\uline{\sffamily{\large التمرين الثاني:(12ن)}}}
	
\begin{itemize}
	\item 
	جسيم يتحرك بسرعة 0.5c كما يشاهده قائد مركبة فضائية تسير بسرعة 0.7c  بالنسبة للأرض.

	
	
	\begin{enumerate}
				
		\item 
	ماهي سرعة الجسم بالنسبة للمشاهد على اﻷرض وفقا لتحويلات غاليليو؟
		\item
	ماهي سرعة الجسم بالنسبة للمشاهد على اﻷرض وفقا للنسبية الخاصة؟
				
		\end{enumerate}	
	\item
	نعرف الكمية اللاتغيرية بالعلاقة التالية:\\	
\begin{center}
		$ (\Delta S)^{2} = (\Delta x)^{2} +(\Delta y)^{2} +(\Delta z)^{2} -c^{2}((\Delta t)^{2}) $
\end{center}
	
حيث $ \Delta S $ يمثل الإحداثي الزمكاني. 
	
	\begin{enumerate}
		
		\item 
		أوجد الزمن الصحيح أو زمن الكون 	$ \Delta\tau $ ؟
		\item
		أوجد المركبة الرابعة لكل من السرعة و الزخم ؟
		\item
		إستنتج علاقة الكتلة النسبية ؟
	\end{enumerate}	
\item
كان السؤال من أين جاءت االزيادة في الكتلة ؟ اﻷثر  الكبير في علاقة الطاقة التي إشتهرى بها آينشتاين. 
	\begin{enumerate}
	
	\item 
بين كيف استنتج آينشتاين ذلك؟
	\item
إستنتج من ذالك علاقة دبرولي  $ p=\dfrac{h}{\lambda} $ ؟
	
   \end{enumerate}	

\end{itemize}	
~~~~~~	
	\\
	\\
	\\
	\\
	\\
	\\
	\\
	\\
	\\
	\\
	\\
	\\
	\\
	\\
	\\
	\\
	\\
	\\
	\\
	\\
	\\
	\\
	
	\begin{flushleft}
		بــالتوفيــق للجميـــع\\
		الأستاذ: د. عميري
		
	\end{flushleft}
	
\end{document}